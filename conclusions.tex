\chapter{Conclusions}

This thesis has addressed the challenges posed by the \acl{HL-LHC} era at \acs{CERN} with a focus on the \acs{FELIX} readout system. The thesis has introduced the \acs{FELIX} project and has presented the contributions to the open-source software \texttt{felix-distribution}, the network library \texttt{netio3}, and the hardware testing and benchmarking of the FLX-182-1B cards. The benchmarks performed at the end have demonstrated that the \acs{FELIX} platform is pushing the limits of the available technology. These results validate the proposed solutions and confirm their suitability for future deployment in large-scale \acs{HEP} experiments. The \acs{FELIX} project is being noted and taken under consideration by the other large experiments at \acs{CERN} thanks to the collaboration and work of all the people involved.

\section{\acs{FELIX} in other experiments}
\begin{itemize}
    \item \textbf{ProtoDUNE SP} \cite{protodune} is an experiment at \acs{CERN} whose purpose is to test and validate the technologies and design that will be applied to the construction of the \acs{DUNE} Far Detector at the Sanford Underground Research Facility. \acs{FELIX} is used for the readout.

    \item \textbf{NA62} \cite{protodune, Martellotti:2056863} is an experiment at \acs{CERN} that studies kaon decays. The \acs{FELIX} platform is used for data buffering, clock distribution and synchronous communication for control and configuration.

    \item \textbf{sPHENIX} \cite{sphenix} is an experiment located at \acl{BNL}, Long Island (New York), whose purpose is to learn about Quark-Gluon Plasma. \acs{FELIX} is used for the readout of three subdetectors.

    \item \textbf{SPIDR4} \cite{spidr4} is a readout system firmware, developed at Nikhef, that uses \acs{FELIX} as a readout system for Timepix4 \cite{timepix4}, a pixel based subdetector. 
\end{itemize}

\section{Future developments}

\subsubsection{felix-tohost improvement}
\label{subsec:felix-tohost-improvement}

In the analysis made with \emph{perf} on \texttt{felix-tohost}, with the FlameGraph results shown in Figure~\ref{fig:felix-tohost-flamegraph}, it is shown that the bottleneck is in decoding \emph{chunks} and \emph{subchunks} from the \emph{blocks} and in the copy operations. An improvement can be made by sending blocks directly instead of decoding them into chunks and then sending a netio3 encoded buffer.\\
Preliminary tests show that \texttt{felix-tohost} performance improves by a factor of 8 (800\%) in the case of 20-byte messages, without much burden on the client side. Further studies are needed to account for all possibilities, for example for chunks that are bigger than a 1KB block and what impact that would have on the performance on both server and client applications.

\subsubsection{cmem\_rcc replacement}

cmem\_rcc has been a very valid option during the years, but the reason it exists is because the Linux kernel could not offer a valid alternative. The last feasibility study is from 2018, when AlmaLinux (The official \acs{CERN} OS) used kernel 3.X, and the kernel 4.X was actively being developed. With kernel 6.X, \acs{DMA} and \acs{NUMA} support for it will be the default, making it an interesting option to avoid using custom kernel modules. At the moment (2025) AlmaLinux 9 is the official release, and it uses kernel 5.X. This replacement can be done only when AlmaLinux 10 with kernel 6.X is released, and supposedly it will happen in time to test the entire system before Phase II.

\subsubsection{Alternative architecture}

It is being taken into consideration to replace the current architecture with one that removes the networking between the \acs{FE} electronics and the data handler. This will reduce the flexibility of the architecture but is counterbalanced with many positive effects like cost savings, space savings, computing power, and energy consumption. In this architecture, the \acl{DH} software will be moved inside the \acs{FELIX} host server, which will need a more powerful CPU to sustain the newly added workloads, but this saves the cost of dedicated \acs{DH} servers. With this new arrangement, the \acs{DH} processes will read data directly from the \acs{FELIX} ToHost \acs{DMA} buffers. This requires software development to allow the data handler to use \acs{RDMA}.

\subsubsection{Using servers with ARM processors}

In an effort to reduce energy consumption at \acs{CERN}, it is being investigated whether ARM processors could be a valid alternative to x86 ones. It will require software adjustments, in particular in the \acs{FELIX} card driver side and kernel compilation.