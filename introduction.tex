\chapter{Introduction}

This document is the result of six months of work at \acs{CERN}, in the \acs{FELIX} project, as part of the hardware and software upgrades for the \acl{HL-LHC} era.

\section{Motivation}

Modern \acl{HEP} experiments, such as those conducted at \acf{CERN}, produce massive quantities of data, necessitating \acf{DAQ} systems capable of high throughput, high scalability, and low latency. As the \acf{LHC} transitions to its High-Luminosity phase \acs{HL-LHC}, the demands on \acs{DAQ} infrastructure have increased, creating the need for new hardware and software solutions. This thesis addresses these challenges by focusing on the readout part inside a \acl{DAQ} system, specifically, the \acs{FELIX} platform and the network library \texttt{netio3}.

\section{Objectives}

The primary objective of this thesis is to contribute to the development of new features of the \acs{FELIX} readout system, the enhancement and optimization of existing ones, testing and assembly of the hardware, and a performance evaluation of the entire system ensuring it meets the requirements of the \acs{HL-LHC} era.

\section{Thesis Structure}

\begin{itemize}
    \item \textbf{Chapter 2:} This chapter provides an overview of \acs{CERN}, its experiments and the role of \acs{FELIX}; it presents where we currently are and better details what will be the future requirements.
    
    \item \textbf{Chapter 3:} This chapter dives into the \acs{FELIX} project in its entirety, detailing the hardware, firmware and software components. At the end of the chapter are the personal contributions to the topic.
    
    \item \textbf{Chapter 4:} This chapter focuses on the \texttt{felix-star} software suite, detailing its most important components. It also includes the personal contributions to the topic.
    
    \item \textbf{Chapter 5:} This chapter presents the \texttt{netio3} network library, making a distinction between the backend and frontend part. Personal contributions end the chapter.
    
    \item \textbf{Chapter 6:} This chapter contains the results of the hardware testing and benchmarking of the FLX-182-1B cards, and the performance evaluation of software.
    
    \item \textbf{Chapter 7:} This chapter concludes the thesis with a summary of the work done and future developments.
\end{itemize}