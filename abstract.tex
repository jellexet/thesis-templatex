\chapter*{Abstract}

\acl{HEP} experiments conducted at \acs{CERN} will require improvements to the data acquisition systems to meet the challenges posed by the High-Luminosity Large Hadron Collider (\acs{HL-LHC}) era. The thesis introduces the \acs{FELIX} project, detailing its role within the \acs{ATLAS} experiment and the \acl{T-DAQ} infrastructure. It provides an overview of the hardware, firmware, and software developments, emphasizing the context for the personal contributions.\\
Personal contributions include the development of software for the new \acl{ITk} Detector in \acs{ATLAS}, used to perform the reconfiguration of radiation-affected electronics. Additionally, a monitoring system was developed to track the \acs{FELIX} software. The work also involved testing and assembling prototype hardware developed and built for the \acl{HL-LHC}.

The results of tests and benchmarks have been performed to validate the proposed solutions. The thesis concludes with an overview of ongoing research and future developments.

\clearpage
\chapter*{Prefazione}

Gli esperimenti di fisica ad alte energie condotti al \acs{CERN} richiederanno miglioramenti nei sistemi di acquisizione dati per affrontare le sfide poste dall'era ad alta luminosità dell'acceleratore di particelle \acs{LHC}.\\
La tesi introduce il progetto \acs{FELIX}, descrivendo il suo ruolo all'interno dell'esperimento \acs{ATLAS} e dell'infrastruttura di Trigger e Acquisizione Dati. Fornisce una panoramica degli sviluppi hardware, firmware e software, enfatizzando il contesto dei contributi personali.\\
I contributi personali includono lo sviluppo di software per il nuovo rivelatore \acl{ITk} nell'esperimento \acs{ATLAS}, utilizzato per la riconfigurazione dell'elettronica affetta da radiazioni. Inoltre, è stato sviluppato un sistema di monitoraggio per tracciare il software \acs{FELIX}. Il lavoro ha anche coinvolto il collaudo e l'assemblaggio di prototipi hardware sviluppati e costruiti per l'\acl{HL-LHC}.

I risultati di test e benchmark sono stati eseguiti per validare le soluzioni proposte. La tesi si conclude con una panoramica delle ricerche in corso e degli sviluppi futuri.